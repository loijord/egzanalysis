\documentclass[a4paper]{article}
\usepackage[utf8]{inputenc}
\usepackage[L7x]{fontenc}
\usepackage[lithuanian]{babel}
\usepackage{lmodern}
\usepackage{graphicx}
\usepackage{amsmath}
\usepackage{animate}
\usepackage{verbatim}
\usepackage{pgfplots}
\usepackage{hyperref}
\usepackage{empheq}
\usepackage[many]{tcolorbox}
\usepackage{mdframed}
\usepackage{framed}
\usepackage{indentfirst}
\usetikzlibrary{arrows} %geogebra
\usepackage[top=2cm, bottom=2cm, left=2cm, right=2cm, footskip=1cm, a4paper]{geometry}

\usepackage{tasks}
\usepackage{hyperref}
\usepackage{graphicx}

\newcommand{\tbf}[1]{\textbf{#1}}
\newcommand{\tit}[1]{\textit{#1}}

\begin{document}
\section{Vektoriai}
\subsection{Ankstesnės temos}
Prieš mokydamiesi šią temą, įsitikinkite, kad turite pakankamai žinių iš ankstesnių geometrinių temų. Pabandykite šiuos uždavinius:
\begin{enumerate}

\begin{minipage}[b]{0.6\linewidth}
\item Stačiojo trikampio įžambinės galų koordinatės yra (1; 2) ir (3; 5).
Remdamiesi brėžiniu nustatykite trečiosios trikampio viršūnės koordinates (a; b).
\tbf{\tit{VBE 2011, 2užd. (1tšk)}}
\end{minipage}
\hspace{\fill} \begin{minipage}{0.23\linewidth} \includegraphics[width=\textwidth]{"vbe2011_2".png}\end{minipage}
\item Įrodykite, kad trikampyje, turinčiame kampus $(90^o, 30^o, 60^o)$, kraštinės visada sutinka santykiu $1:2:\frac{\sqrt{3}}{2}$
\item Įrodykite, kad trikampyje, turinčiame kampus $(90^o,X^o, (90-X)^o)$, kraštinės visada sutinka santykiu $(1, \sin (X^o), \cos (X^o))$
\end{enumerate}
\subsection{Ilgis ir kampas atskirai (apžvalga, ką jau turime mokėti)}
Ankstesnėse temose jums yra tekę pasinaudojant įvairiomis savybėmis ir teoremomis susieti atkarpų ilgius ir kampus. Iš pat pradžių uždaviniams išspręsti užtekdavo šių taisyklių:
\begin{itemize} 
\item Pitagoro teoremos, kuri parodo sąryšį tarp trijų kraštinių, kai žinome, kad kampas tarp kažkurių dviejų lygus $90^o$.
\item Lygiašonio trikampio savybe, kad 2 lygios kraštinės jame bus tada ir tik tada, kai bus 2 lygūs kampai. 
\end {itemize}
Vėliau būdavo galima išvesti dar vieną taisyklę:
\begin{itemize} 
\item Trikampių, turinčių kampus $(90^o, 30^o, 60^o)$, kraštinės visada sutinka santykiu $1:2:\frac{\sqrt{3}}{2}$, tačiau ši taisyklė tėra tik ankstesnių dviejų savybių hibridas, nes šį teiginį galima įrodyti tada ir tik tada, kai taikome Pitagoro teoremą ir pastebime, kad trikampis su kampais $(90^o, 30^o, 60^o)$ yra lygiakraščio trikampio pusė.
\end{itemize}
Dar vėliau domėjomės, kokias kraštinių proporcijas išlaiko šįkart jau bet kurie trikampiai su kampais $(90^o,X^o, (90-X)^o)$ ir radome atsakymą:
\begin{itemize}
\item Trikampio su kampais $(90^o,X^o, (90-X)^o)$ kraštinės sutinka santykiu $(1, \sin (X^o), \cos (X^o))$. 
\end{itemize}
Kai jau pakankamai susipažinome, kokie yra sąryšiai tarp atkarpų ilgių ir kampų, metas patyrinėti geometrinius objektus, kuriems aprašyti reikia ir ilgio, ir kampo kartu.
\subsection{Ilgis ir kampas viename. Kas yra vektorius ir kam jį panaudoti?}

\begin{empheq}[box=\tcbhighmath]{align*}
\text{\tbf{Geometrinė interpretacija}}
\end{empheq}

\tbf{Vektorius yra geometrinis objektas, turintis kryptį ir ilgį.} Jis visada gali būti vaizduojamas kaip rodyklė.

\begin{empheq}[box=\tcbhighmath]{align*}
\text{\tbf{Algebrinė interpretacija}}
\end{empheq}

\tbf{Vektorius} (\tit{lot. vector - vežikas}) \tbf{apibrėžia, kokiomis koordinatėmis reikia paslinkti tašką $A$, kad jis patektų į tašką $B$}. Vadinasi, jis gali būti aprašomas ir kaip koordinačių seka. Pvz. $(x, y)$ yra plokštumos vektorius, o $(x,y,z)$ yra trimatės erdvės vektorius.\par
Vadinasi, vektoriai gali būti panaudoti spręsti geometriniams uždaviniams, kuriuose yra patogiau naudoti koordinates, nei tik kampus ir atkarpas. Toliau mokysimės, kaip įvairias operacijas su vektoriais aprašyti algebriškai, t.y. naudojant koordinačių sekas.

\subsection{Vektorių operacijos}

Veiksmai, kuriuos galima atlikti su vektoriais (\tbf{juos privaloma mokėti egzaminui}):
\begin{mdframed}[backgroundcolor=yellow!50!white]
\begin{itemize}
\item Apskaičiuoti vektoriaus $\overrightarrow{x}$ ilgį $|\overrightarrow{x}|$. 
\item Apkeisti vektoriaus $\overrightarrow{x}$ kryptį padauginę jo koordinates iš -1.
\item Padauginti vektorių $\overrightarrow{x}$ iš skaliaro $k$ (bet kurio nenulinio realiojo skaičiaus) ir gauti naują $k$ kartų ilgesnį vektorių $k\overrightarrow{x}$, išlaikantį tą pačią kryptį, jei $k>0$ ir priešingos krypties, jei $k<0$
\item Sudėti du vektorius $\overrightarrow{x}$ ir $\overrightarrow{y}$ ir gauti naują vektorių $\overrightarrow{x} + \overrightarrow{y}$, kurio ilgis ir kryptis sutampa su lygiagretainio, kurio kraštinės yra vektoriuose $\overrightarrow{x}$ ir $\overrightarrow{y}$, įstrižaine.
\item Atimti du vektorius $\overrightarrow{x}$ ir $\overrightarrow{y}$ ir gauti naują vektorių, lygų vektorių $\overrightarrow{x}$ ir $-\overrightarrow{y}$ sumai.
\item Sudauginti du vektorius  $\overrightarrow{x}$ ir $\overrightarrow{y}$ skaliariškai ir gauti jų skaliarinę sandaugą $\overrightarrow{x} \cdot \overrightarrow{y}$.
\item Patikrinti, ar du vektoriai $\overrightarrow{x}$ ir $\overrightarrow{y}$ statmeni: jie statmeni tada ir tik tada, kai $\overrightarrow{x} \cdot \overrightarrow{y}=0$
\item Patikrinti, ar du vektoriai $\overrightarrow{x} \cdot \overrightarrow{y}$ kolinearūs (\tit{angl. co - bendras, line - tiesė}), t.y. ar jie gali būti pavaizduoti vienoje tiesėje: tai įmanoma tada ir tik tada, jeigu egzistuoja skaliaras (nenulinis) $k$, kad $\overrightarrow{x}= k\cdot \overrightarrow{y}$
\end{itemize}
\end{mdframed}
\subsection{Uždaviniai, kurie padės susieti geometrinę ir algebrinę interpretacijas}

\begin{enumerate}
\item Koordinačių plokštumoje pažymėkite bet kuriuos du taškus $A(x_1,y_1)$ ir $B(x_2, y_2)$. Nubrėžkite rodyklę iš taško $A$ į tašką $B$, kuri atitinka vektorių $\overrightarrow{AB}$. Remkitės \fbox{algebrine vektoriaus interpretacija} ir grafiškai parodykite, kad $\overrightarrow{AB}=(x_2-x_1, y_2-y_1)$.
\item Kodėl vektoriaus $(x,y)$ ilgis lygus $\sqrt{x^2+y^2}$? Į šį klausimą galite atsakyti koordinačių plokštumoje pažymėję taškus $A(0,0)$ ir $B(x, y)$ ir parodydami, kad vektoriaus $\overrightarrow{AB}$ ilgis lygus $\sqrt{x^2+y^2}$
\item Koordinačių plokštumoje pažymėkite taškus $A(0,0)$, $B(x, y)$, $C(kx, ky)$. $\overrightarrow{AB}$. Įsitikinkite, kad $k\cdot \overrightarrow{AB}=\overrightarrow{AC}$
\item Įsitikinkite, kad bet kuriam trikampiui $ABC$, esančiam ant koordinačių plokštumos, galioja: 
\begin{enumerate}
\item $\overrightarrow{AB}+\overrightarrow{BC}=\overrightarrow{AC}$
\item $\overrightarrow{AB}+\overrightarrow{BC}+\overrightarrow{CA}=(0, 0)$
\end{enumerate}
\item Koordinačių plokštumoje pažymėkite du vektorius $\overrightarrow{AB}=(x_1, y_1)$ ir $\overrightarrow{BC}=(x_2, y_2)$, taip, kad $\overrightarrow{AB}$ galas sutaptų su $\overrightarrow{BC}$ pradžia ir įsitikinkite, kad vektorių $(x_1, y_1)$ ir $(x_2, y_2)$ suma lygi $(x_1+x_2, y_1+y_2)$
\item Duota, kad kampas tarp vektorių $\overrightarrow{AB}$ ir $\overrightarrow{BC}$ yra lygus $67^o$. Kam lygus kampas tarp vektorių $\overrightarrow{AB}$ ir $-\overrightarrow{BC}$?
\item Vektorių $\overrightarrow{u}=(x_1, y_1)$ ir $\overrightarrow{v}=(x_2, y_2)$ skaliarinė sandauga $\overrightarrow{u}\cdot \overrightarrow{v}$ visuomet lygi $x_1y_1+x_2y_2$. Įrodykite, kad kiekvienam dvimačiui vektoriui $\overrightarrow{a}=(x, y)$ visuomet galioja $|\overrightarrow{a}|^2=\overrightarrow{a}\cdot \overrightarrow{a}$. Ar ši tapatybė teisinga trimačiams vektoriams?
\end{enumerate}
%\item Išsiaiškinsime dviejų vektorių $\overrightarrow{u}=(x_1, y_1)$ ir $\overrightarrow{v}=(x_2, y_2)$ skaliarinės %sandaugos $\overrightarrow{u} \cdot \overrightarrow{v} = (x_1x_2, y_1y_2)$ geometrinę prasmę
\subsection{Pora tapatybių, kurias reikia mokėti egzaminui}

Egzaminų lape pateikiami du skaliarinės sandaugos apibrėžimai (tai yra viskas, ką pateikia iš vektorių temos):
\begin{framed}
\noindent\includegraphics[width=\textwidth]{"egz_vektoriai".png}
\end{framed}

Jais pasinaudojus galima gauti, kad kampas $\alpha$, kurį sudaro vektoriai $\overrightarrow{u}=(x_1,y_1,z_1)$ ir $\overrightarrow{v}=(x_2,y_2,z_2)$ tenkina

\begin{empheq}[box=\tcbhighmath]{align*}
\cos \alpha = \frac{\overrightarrow{u}\cdot \overrightarrow{v}}{|\overrightarrow{u}|\cdot |\overrightarrow{v}|}
\end{empheq}

Kitaip tariant, to kampo kosinusas lygus vektorių $\overrightarrow{u}$ ir $\overrightarrow{v}$ ilgių sandaugai, padalintai iš skaliarinės sandaugos. Ar sugebėtumėte rasti kampą tarp vektorių $\overrightarrow{u}=(1,0,1)$ ir $\overrightarrow{v}=(1,1,1)$?

Lieka neaišku, kodėl skaliarinę sandaugą galima aprašyti dviem būdais. Pasirodo, tai yra kosinusų teoremos išvada. Įrodymą galite išsinagrinėti \href{http://clas.sa.ucsb.edu/staff/alex/DotProductDerivation.pdf}{šioje nuorodoje}

Dar viena tapatybė, kuri pasitaiko egzaminuose:
\begin{empheq}[box=\tcbhighmath]{align*}
|\overrightarrow{a}|^2=\overrightarrow{a}\cdot \overrightarrow{a}
\end{empheq}
\subsection{Ruošiamės egzaminui}
\begin{enumerate}
%\item \tbf{\tit{VBE 2008, 17užd. (3tšk)}}\\
\item \tbf{\tit{VBE 2009, 2užd. (1tšk)}}\\
Jei lygiakraščio trikampio $ABC$ kraštinės ilgis lygus 4, tai kam lygi skaliarinė sandauga $\overrightarrow{BA}\cdot \overrightarrow{BC}$?
\item \tbf{\tit{VBE 2010, 19užd. (1tšk,2tšk,2tšk)}}\\
\begin{minipage}[b]{0.6\linewidth}
Koordinačių plokštumoje duoti trys taškai $A(3;6)$, $B(6; 12)$ ir $C(13;1)$.
\begin{enumerate}
\item Užrašykite vektoriaus $\overrightarrow{AB}$ koordinates.
\item Ar vektoriai $\overrightarrow{AB}$ ir $\overrightarrow{AC}$ statmeni?
\item Toje pačioje koordinačių plokštumoje pasirinktas taškas $D$, kad keturkampis $ABCD$ būtų lygiagretainis. Nustatykite taško $D$ koordinates.
\end{enumerate}
\end{minipage}
\hspace{\fill}
\definecolor{uuuuuu}{rgb}{0.26666666666666666,0.26666666666666666,0.26666666666666666}
\definecolor{qqqqff}{rgb}{0.,0.,1.}
\definecolor{cqcqcq}{rgb}{0.7529411764705882,0.7529411764705882,0.7529411764705882}
\begin{tikzpicture}[line cap=round,line join=round,>=triangle 45,x=1.0cm,y=1.0cm, scale=0.3]
\draw [color=cqcqcq,, xstep=2.0cm,ystep=2.0cm] (-0.595,-0.6075) grid (16.945,12.5325);
\draw[->,color=black] (-0.595,0.) -- (16.945,0.);
\foreach \x in {2,4,6,8,10,12,14,16}
\draw[shift={(\x,0)},color=black] (0pt,2pt) -- (0pt,-2pt) node[below] {\footnotesize $\x$};
\draw[->,color=black] (0.,-0.6075) -- (0.,12.9325);
\foreach \y in {2,4,6,8,10,12}
\draw[shift={(0,\y)},color=black] (2pt,0pt) -- (-2pt,0pt) node[left] {\footnotesize $\y$};
\draw[color=black] (0pt,-10pt) node[right] {\footnotesize $0$};
\clip(-0.595,-0.6075) rectangle (16.945,12.9325);
\draw (13.,1.)-- (16.,7.);
\draw (16.,7.)-- (6.,12.);
\draw [->] (3.,6.) -- (6.,12.);
\draw [->] (3.,6.) -- (13.,1.);
\draw [fill=qqqqff] (3.,6.) circle (2.5pt);
\draw[color=qqqqff] (3.215,6.5475) node {$A$};
\draw [fill=qqqqff] (6.,12.) circle (2.5pt);
\draw[color=qqqqff] (6.215,12.4275) node {$B$};
\draw [fill=qqqqff] (13.,1.) circle (2.5pt);
\draw[color=qqqqff] (13.205,1.5675) node {$C$};
\draw [fill=uuuuuu] (16.,7.) circle (1.5pt);
\draw[color=uuuuuu] (16.175,7.4475) node {$D$};
\end{tikzpicture}
\item  \tbf{\tit{VBE 2012, 24užd. (2tšk)}}\\
Su kuria $x$ reikšme vektoriai $\overrightarrow{c}=(x-5)\overrightarrow{i}+\overrightarrow{j}$ ir $\overrightarrow{c}=(2x-1)\overrightarrow{i}-\overrightarrow{j}$ yra kolinearūs? ($\overrightarrow{i}$ ir $\overrightarrow{j}$ yra vienetiniai vektoriai koordinačių ašyje, lygūs $(1,0)$ ir $(0, 1)$)
\item  \tbf{\tit{VBE 2013, 20užd. (2tšk)}}\\
Vektoriai $\overrightarrow{a}-2\overrightarrow{b}$ ir $\overrightarrow{a}+2\overrightarrow{b}$ statmeni, $|\overrightarrow{a}|=5$. Raskite $|\overrightarrow{b}|$.
\item  \tbf{\tit{VBE 2014, 16užd. (2tšk)}}\\
Duoti taškai $A(-1; -2; 4), B(-4; -2; 0), C(3; -2;1)$. Apskaičiuokite kampo tarp vektorių $\overrightarrow{BA}$ ir $\overrightarrow{BC}$ didumą.
\item \tbf{\tit{VBE 2014, 28užd. (4tšk)}}\\
\begin{minipage}[c]{0.5\linewidth}
Taškai $K$ ir $M$ yra lygiagretainio $ABCD$ kraštinių $BC$ ir $CD$ vidurio taškai. Vektorių
$\overrightarrow{AD}$ išreikškite vektoriais $\overrightarrow{AM}=\overrightarrow{a}$ ir $\overrightarrow{AK}= \overrightarrow{b}$.
\end{minipage}
\hspace{\fill} \begin{minipage}{0.4\linewidth} \includegraphics[width=\textwidth]{"vbe2014_28".png}\end{minipage}
\begin{minipage}[b]{0.7\linewidth}
\item  \tbf{\tit{VBE 2014band, 17užd. (2tšk)}}\\
Raskite vektoriaus $\overrightarrow{c}$ ilgį, jei $\overrightarrow{c}=2\overrightarrow{a}-3\overrightarrow{b}$ ir $\overrightarrow{a}=(0; 0,5)$, $\overrightarrow{b}=(-2; 3)$.
\item  \tbf{\tit{VBE 2015band, 16užd. (2tšk, 2tšk)}}\\
Keturkampis $ABCD$ yra rombas. 
\begin{enumerate}
\item Užrašykite vektorių, lygų vektorių sumai $\overrightarrow{AB}+\overrightarrow{AD}$.
\item Apskaičiuokite vektorių skaliarinę sandaugą $\overrightarrow{BD}\cdot \overrightarrow{AC}$
\end{enumerate}
\end{minipage}
\hspace{\fill} \begin{minipage}{0.15\linewidth} \includegraphics[width=\textwidth]{"vbe2015_16".png}\end{minipage}
\item \tbf{\tit{VBE 2015, 7užd. (2tšk)}}\\
Su kuria $x$ reikšme vektoriai $\overrightarrow{a}=(x; 3)\text{ ir }b=(-2; 6)$ yra kolinearūs?
\end{enumerate}
\end{document}