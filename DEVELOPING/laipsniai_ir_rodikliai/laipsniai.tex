\documentclass[a4paper]{article}
\usepackage[utf8]{inputenc}
\usepackage[L7x]{fontenc}
\usepackage[lithuanian]{babel}
\usepackage{lmodern}
\usepackage{amsmath}
\usepackage[top=2cm, bottom=2cm, left=2cm, right=2cm, footskip=1cm, a4paper]{geometry}

\begin{document}
\begin{enumerate}
%\item Panaudokite formulę $\boxed{\sin \alpha \cos \beta=\frac{1}{2}\left(\sin (\alpha-\beta)+\sin (\alpha+\beta)\right)}$ reiškiniui $2\sin(4\alpha)\cos(2\alpha) - \sin(6\alpha)$ suprastinti.
\item VBE 2018[1]. Išspręskite lygtį $4^{x-3}\cdot 2^{x-4}=0$
\item VBE 2017[1]. Skaičių $\sqrt[3]{2017\sqrt[3]{2017}}$ išreikškite 2017 laipsniu.
\item VBE 2017[1]. Išspręskite lygtį $2^{x+3}=16$
\item VBE 2017[1]. Išspręskite lygtį $\sqrt{2-x}=3$
\item VBE 2016[1]. Išspręskite lygtį $9^{x+1}=3^{4x-2}$
\item VBE 2013[1]. Visus iš eilės einančius natūraliuosius skaičius keliant kvadratu buvo gauta seka $1^2, 2^2, 3^2, \dots n^2, \dots$. Skaičius $10^8$ yra šios sekos narys. Kuris skaičius šioje sekoje eis iš karto po jo?
\end{enumerate}
\end{document}